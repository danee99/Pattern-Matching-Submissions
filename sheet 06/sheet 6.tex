\documentclass[12pt]{article}
\usepackage[utf8]{inputenc}
\usepackage[automark]{scrlayer-scrpage}
\usepackage[T1]{fontenc}
\usepackage[pdftex]{graphicx,color}
\usepackage{subcaption}
\usepackage{url}
\usepackage[pdftex,pdfpagelabels]{hyperref}
\usepackage[section,boxed]{algorithm}
\usepackage{algpseudocode}
\usepackage{amsmath}
\usepackage{amssymb}
\usepackage{amsthm}
\usepackage{lmodern}
\pagestyle{scrheadings}
\ofoot{}\cfoot{}\ifoot{}
\lehead{\pagemark}
\rehead{\slshape\leftmark}
\lohead{\slshape\leftmark}
\rohead{\pagemark}
\newtheorem{definition}{Definition}
\newtheorem{theorem}{Theorem}
\usepackage{blindtext}
\usepackage[ngerman]{babel}
\usepackage[margin=1in]{geometry} 
\usepackage{amsmath,amsthm,amssymb}
\usepackage{graphicx}
\newcommand{\N}{\mathbb{N}}
\newcommand{\Z}{\mathbb{Z}}
\newcommand{\R}{\mathbb{R}}
\newcommand{\C}{\mathbb{C}}
 
\begin{document}
% --------------------------------------------------------------
%                         Title Page
% --------------------------------------------------------------
\title{Solutions for Sheet 6}
\author{Raphael Wude, Martin Brückmann, Claude Jordan, Daniel Degenstein\\ \\
\textsc{Pattern Matching and Machine Learning} \\
\textsc{for Audio Signal Processing}}
\maketitle

% --------------------------------------------------------------
%                         Task 6.1
% --------------------------------------------------------------
\section*{Task 6.1}
The time-shift operator $T^{k}$, $k \in \mathbb{Z}$ is defined as $T^{k}[x](n) = x^{k}(n) := x(n-k)$.\\\\
A system $S$ is time invariant $\Leftrightarrow \forall k \in \mathbb{Z}, \forall x \in \ell^{p}(\mathbb{Z}): S[x^{k}]=S[x]^{k}$.

\subsection*{(a)}
The upsampling operator is defined as (${\uparrow}M)[x](n)=\Bigg\{\begin{array}{l} x(n/M), ~~~~~\text{if } M \text{ divides } n\text{,}\\ 0, ~~~~~~~~~~~~~~\text{otherwise.} \end{array}$\\\\\\
$\Rightarrow$ (${\uparrow}M)[x^{k}](n)=\Bigg\{\begin{array}{l} x^{k}(n/M), ~~~~~\text{if } M \text{ divides } n\text{,}\\ 0, ~~~~~~~~~~~~~~~\text{otherwise.} \end{array}$\\\\\\
$=\Bigg\{\begin{array}{l} x(n/M -k), ~~~~~\text{if } M \text{ divides } n\text{,}\\ 0, ~~~~~~~~~~~~~~~~~~~\text{otherwise.} \end{array}$\\\\\\
$\neq$ (${\uparrow}M)[x]^{k}(n)=$(${\uparrow}M)[x](n-k)=\Bigg\{\begin{array}{l} x((n-k)/M), ~~~~~\text{if } M \text{ divides } (n-k)\text{,}\\ 0, ~~~~~~~~~~~~~~~~~~~~~\text{otherwise.} \end{array}$\\\\\\
$\Rightarrow$ The upsampling operator is not time invariant.
\newpage

\subsection*{(b)}
The frequency-shift operator is defined as $E_{w}[x](n):=e^{-2\pi iwn}x(n)$.\\\\
$\Rightarrow E_{w}[x^{k}](n)=e^{-2\pi iwn}x^{k}(n) = e^{-2\pi iwn}x(n-k) \neq e^{-2\pi iw(n-k)}x(n-k) =$\\ $E_{w}[x](n-k) = E_{w}[x]^{k}(n)$\\\\
$\Rightarrow$ The frequency-shift operator is not time invariant.
 
\end{document}