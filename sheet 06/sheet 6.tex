\documentclass[12pt]{article}
\usepackage[utf8]{inputenc}
\usepackage[automark]{scrlayer-scrpage}
\usepackage[T1]{fontenc}
\usepackage[pdftex]{graphicx,color}
\usepackage{subcaption}
\usepackage{url}
\usepackage[pdftex,pdfpagelabels]{hyperref}
\usepackage[section,boxed]{algorithm}
\usepackage{algpseudocode}
\usepackage{amsmath}
\usepackage{amssymb}
\usepackage{amsthm}
\usepackage{lmodern}
\pagestyle{scrheadings}
\ofoot{}\cfoot{}\ifoot{}
\lehead{\pagemark}
\rehead{\slshape\leftmark}
\lohead{\slshape\leftmark}
\rohead{\pagemark}
\newtheorem{definition}{Definition}
\newtheorem{theorem}{Theorem}
\usepackage{blindtext}
\usepackage[ngerman]{babel}
\usepackage[margin=1in]{geometry} 
\usepackage{amsmath,amsthm,amssymb}
\usepackage{graphicx}
\newcommand{\N}{\mathbb{N}}
\newcommand{\Z}{\mathbb{Z}}
\newcommand{\R}{\mathbb{R}}
\newcommand{\C}{\mathbb{C}}
 
\begin{document}
% --------------------------------------------------------------
%                         Title Page
% --------------------------------------------------------------
\title{Solutions for Sheet 6}
\author{Raphael Wude, Martin Brückmann, Claude Jordan, Daniel Degenstein\\ \\
\textsc{Pattern Matching and Machine Learning} \\
\textsc{for Audio Signal Processing}}
\maketitle

% --------------------------------------------------------------
%                         Task 6.1
% --------------------------------------------------------------
\section*{Task 6.1}
The time-shift operator $T^{k}$, $k \in \mathbb{Z}$ is defined as $T^{k}[x](n) = x^{k}(n) := x(n-k)$.\\\\
A system $S$ is time invariant $\Leftrightarrow \forall k \in \mathbb{Z}, \forall x \in \ell^{p}(\mathbb{Z}): S[x^{k}]=S[x]^{k}$.

\subsection*{(a)}
The upsampling operator is defined as (${\uparrow}M)[x](n)=\Bigg\{\begin{array}{l} x(n/M), ~~~~~\text{if } M \text{ divides } n\text{,}\\ 0, ~~~~~~~~~~~~~~\text{otherwise.} \end{array}$\\\\\\
$\Rightarrow$ (${\uparrow}M)[x^{k}](n)=\Bigg\{\begin{array}{l} x^{k}(n/M), ~~~~~\text{if } M \text{ divides } n\text{,}\\ 0, ~~~~~~~~~~~~~~~\text{otherwise.} \end{array}$\\\\\\
$=\Bigg\{\begin{array}{l} x(n/M -k), ~~~~~\text{if } M \text{ divides } n\text{,}\\ 0, ~~~~~~~~~~~~~~~~~~~\text{otherwise.} \end{array}$\\\\\\
$\neq$ (${\uparrow}M)[x]^{k}(n)=$(${\uparrow}M)[x](n-k)=\Bigg\{\begin{array}{l} x((n-k)/M), ~~~~~\text{if } M \text{ divides } (n-k)\text{,}\\ 0, ~~~~~~~~~~~~~~~~~~~~~\text{otherwise.} \end{array}$\\\\\\
$\Rightarrow$ The upsampling operator is not time invariant.
\newpage

\subsection*{(b)}
The frequency-shift operator is defined as $E_{w}[x](n):=e^{-2\pi iwn}x(n)$.\\\\
$\Rightarrow E_{w}[x^{k}](n)=e^{-2\pi iwn}x^{k}(n) = e^{-2\pi iwn}x(n-k) \neq e^{-2\pi iw(n-k)}x(n-k) =$\\ $E_{w}[x](n-k) = E_{w}[x]^{k}(n)$\\\\
$\Rightarrow$ The frequency-shift operator is not time invariant.

\section*{6.2}
\subsection*{(a)}

$C(\mathcal{D}) = \{(1,1), (1,3), (2,4), (3,1), (3,5), (4,2), (5,3),\\
(5,5), (6,1), (6,2), (6,5), (7,4), (8,2), (9,1), (9,4)\}$\\
$C(\mathcal{Q})=\{(1,3), (2,1), (2,5), (3,4), (4,2)\}$

\subsection*{(b)}
The shifted constellation maps are given by

$$m + C(\mathcal{Q}) = \{(m+1,3), (m+2,1),(m+2,5),(m+3,4),(m+4,2)\}$$

for $m \in [-3:8]$, i.e.

$$-3+ C(\mathcal{Q}) = \{(-2,3), (-1,1), (-1,5), (0,4), (1,2)\}$$
$$-2+ C(\mathcal{Q}) = \{(-1,3), (0,1), (0,5), (1,4), (2,2)\}$$
$$\vdots$$
$$7+ C(\mathcal{Q}) = \{(8,3), (9,1), (9,5), (10,4), (11,2)\}$$
$$8+ C(\mathcal{Q}) = \{(9,3), (10,1), (10,5), (11,4), (12,2)\}$$

\subsection*{(c)}
For $m \in \mathbb{Z} \setminus [-3:8]$, the points cannot overlap, so $\Delta_C(m) = 0$. Furthermore, we have

$$\Delta_C(-3) = \Delta_C(-2) = 0$$
$$\Delta_C(-1) = |\{(1,1), (2,4)\}| = 2$$
$$\Delta_C(0) = |\{(1,3), (4,2)\}| = 2$$
$$\Delta_C(1) = |\{(3,1), (3,5)\}| = 2$$
$$\Delta_C(2) = |\{(6,2)\}| = 1$$
$$\Delta_C(3) = |\{(5,5)\}| = 1$$
$$\Delta_C(4) = |\{(5,3), (6,1), (6,5), (7,4), (8,2)\}| = 5$$
$$\Delta_C(5) = 0$$
$$\Delta_C(6) = |\{(9,4)\}| = 1$$
$$\Delta_C(7) = |\{(9,1)\}| = 1$$
$$\Delta_C(8) = 0 $$



\end{document}