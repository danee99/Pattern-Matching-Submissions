\documentclass[12pt]{article}
\usepackage[utf8]{inputenc}
\usepackage[automark]{scrlayer-scrpage}
\usepackage[T1]{fontenc}
\usepackage[pdftex]{graphicx,color}
\usepackage{subcaption}
\usepackage{url}
\usepackage[pdftex,pdfpagelabels]{hyperref}
\usepackage[section,boxed]{algorithm}
\usepackage{algpseudocode}
\usepackage{amsmath}
\usepackage{amssymb}
\usepackage{amsthm}
\usepackage{lmodern}
\pagestyle{scrheadings}
\ofoot{}\cfoot{}\ifoot{}
\lehead{\pagemark}
\rehead{\slshape\leftmark}
\lohead{\slshape\leftmark}
\rohead{\pagemark}
\newtheorem{definition}{Definition}
\newtheorem{theorem}{Theorem}
\usepackage{blindtext}
\usepackage[ngerman]{babel}
\usepackage[margin=1in]{geometry}
\usepackage{amsmath,amsthm,amssymb}
\usepackage{graphicx}
\newcommand{\N}{\mathbb{N}}
\newcommand{\Z}{\mathbb{Z}}
\newcommand{\R}{\mathbb{R}}
\newcommand{\C}{\mathbb{C}}
\usepackage{multirow}

\begin{document}
% --------------------------------------------------------------
%                         Title Page
% --------------------------------------------------------------
\title{Solutions for Sheet 7}
\author{Raphael Wude, Martin Brückmann, Claude Jordan, Daniel Degenstein\\ \\
\textsc{Pattern Matching and Machine Learning} \\
\textsc{for Audio Signal Processing}}
\maketitle

% --------------------------------------------------------------
%                         Task 1.1
% --------------------------------------------------------------
\section*{Task 7.1}
\begin{itemize}
    \item[(a) \& b)]
    $C(\mathcal{D}) = \{(1,1), (1,3), (2,4), (3,1), (3,5), (4,2), (5,3),\\
    (5,5), (6,1), (6,2), (6,5), (7,4), (8,2), (9,1), (9,4)\}$\\
    $C(\mathcal{Q})=\{(1,3), (2,1), (2,5), (3,4), (4,2)\}$\\
    For this values we get for the inverted list:
    \begin{align*}
      L(1) &= (1,3,6,9)\\
      L(2) &= (4,6,8)\\
      L(3) &= (1,5)\\
      L(4) &= (2,7,9)\\
      L(5) &= (3,5,6)
    \end{align*}
    So the indicator function and the resulting matching functions are:
    \begin{table}[h]
\centering
\begin{tabular}{c|c|c|c|c|c|c|c|c|c|c}
    \multirow{2}{*}{\textbf{Query}} & \multirow{2}{*}{\textbf{L(h) - n}}& \multicolumn{8}{c}{\textbf{indicator functions}}  \\
     & & -1 & 0 & 1 & 2 & 3 & 4 & 5 & 6 & 7\\
     \hline
     (1,3) & (0,4) & 0&1&0&0&0&1&0&0&0\\
     (2,1) & (-1,1,4,7) & 1&0&1&0&0&1&0&0&1\\
     (2,5) & (1,3,4) & 0&0&1&0&1&1&0&0&0\\
     (3,4) & (-1,4,6) & 1&0&0&0&0&1&0&1&0\\
     (4,2) & (0,2,4) & 0&1&0&1&0&1&0&0&0\\
       \hline
     \multicolumn{2}{c}{$\Delta_F$} &2&2&2&1&1&5&0&1&1
\end{tabular}
\caption{indicator function and matching function for $C(\mathcal{Q})$ and $C(\mathcal{D})$}
\label{tab:ind_func}
\end{table}

\end{itemize}


\section*{Task 7.2}
- $p \in [0,1] =$ probability of a spectral peak to survive in a query audio fragment.\\
- original document contains in each target zone exactly $F$ spectral peaks\\\\
$E_{k} =$ the anchor point and at least $k$ target points survive\\\\
$\Rightarrow P(E_{k}) = p \cdot P(X \geq k) = p \cdot (1-P(X < k)) = p \cdot (1 - \sum_{j=0}^{k-1} P(X = j))$\\\\
We interpret the peak survival for a frame as a set of Bernoulli experiments. We have $F$ spectral peaks. Therefore we obtain the following binomial distribution:\\
\[
P(X = k) = \binom{F}{k}p^{k}(1-p)^{F-k}
\]
We can simplify the formular for $k=1$ and $k=2$.\\\\
$k=1:$\\\\
$\Rightarrow P(E_{1}) = p \cdot (1 - P(X=0))$\\\\
$= p \cdot (1 - \binom{F}{0}p^{0}(1-p)^{F-0})$\\\\
$= p \cdot (1 - 1 \cdot 1 \cdot (1-p)^{F})$\\\\
$= p \cdot (1-(1-p)^{F})$\\\\
$= p-p \cdot (1-p)^{F}$\\\\\\
$k=2$\\\\
$\Rightarrow P(E_{2}) = p \cdot (1-P(X=0) - P(X=1))$\\\\
$= p \cdot (1 - \binom{F}{0}p^{0}(1-p)^{F-0}-\binom{F}{1}p^{1}(1-p)^{F-1})$\\\\
$= p \cdot (1 - 1 \cdot 1 \cdot (1-p)^{F} - F \cdot p(1-p)^{F-1})$\\\\
$= p \cdot (1 - (1 - p)^{F} - F \cdot p(1-p)^{F-1})$\\\\
$= p - p(1-p)^{F} - F \cdot p^{2}(1-p)^{F-1}$\\\\\\
The probability for $p = 0.5$, $F = 11$ and $k = 2$ is the following:\\\\
$P(E_{2}) = p - p(1-p)^{F} - F \cdot p^{2}(1-p)^{F-1}$\\\\
$= 0.5 - 0.5 \cdot 0.5^{11} - 11 \cdot 0.5^{2} \cdot 0.5^{10}$\\\\
$= 0.5 - 0.5^{12} - 11 \cdot 0.5^{12} \approx 0.497$




\end{document}
